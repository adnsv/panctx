\chapter[remote-control]{Remote Control}

A remote is supplied with the unit and allows for operation of the device
wirelessly. Once the software has been initialized, the remote allows the
operator to perform common tasks including probe setup and getting images. When
not in use, the remote is stored in a specialized holder on the back on the
device.

\section[button-assignments]{Button Assignments}

\placefigure[none,right][]{}{\externalfigure[/Users/alex/projects/user-docs/f1-hardware-manual/img/remote-buttons.svg][conversion=mp,height=230.000000pt]}

\startdescription{Setup/Get}
Perform Setup or Get routine for taking a measurement
\stopdescription

\startdescription{+ / -}
Increase/decrease the diameter of the manual circle
\stopdescription

\startdescription{Pass/Fail}
Make a decision on weld quality
\stopdescription

\startdescription{Next}
Move to next weld or unit
\stopdescription

\startdescription{Back}
Exit from menu or dialog
\stopdescription

\startdescription{Ok}
Confirm selection in menus and dialogs
\stopdescription

\startdescription{Up/Down}
Up and down cursor keys; in Array Explorer’s main window: Navigate up and down in the weld list view
\stopdescription

\startdescription{Left/Right}
Left and right cursor keys; in Array Explorer’s main window: Navigate left and right in recent measurements
\stopdescription

\startdescription{A}
Opens drop down menu in A-scan view
\stopdescription

\startdescription{B}
Toggles between front and back cells in weld list view
\stopdescription

\startdescription{C}
Toggles between automatic and manual circle measurements
\stopdescription

\section[pairing]{Pairing}

The pairing procedure will set up the receiver to be controlled by a specific
remote. In the environment with multiple RSWA units and remotes, this allows the
user to choose specifically which remote controls which RSWA.

To pair the remote control with the receiver:

\startitemize
\item
Make sure the receiver is plugged into the RSWA and that RSWA is running
\item
Bring the remote within a few inches from the receiver located above the
screen, push and hold any button on the remote for 5 seconds. The LED on the
remote will flash rapidly during the pairing procedure;
\item
Release the button and test the remote with RSWA applications.
\stopitemize

If you need to pair a remote to a different RSWA, remove the battery from the
remote for about 1 minute. This will reset it to the original unpaired state.
Then put the battery back and repeat the pairing procedure with the different
receiver.

\section[replacing-the-battery]{Replacing the Battery}

This remote needs a battery to operate. Replace the battery when needed with CR2025 battery.

\placefigure[none][]{}{\externalfigure[/Users/alex/projects/user-docs/f1-hardware-manual/img/remote-battery.svg][conversion=mp,width=0.500000\textwidth]}

To replace the battery:

\startitemize
\item
Pull the battery compartment out and remove the old battery
\item
Insert the replacement battery with the positive side facing up
\item
Close the battery compartment
\item
If necessary, perform the pairing procedure
\stopitemize